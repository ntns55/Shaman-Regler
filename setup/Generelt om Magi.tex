\section*{Generelle regler om Magi}
\addcontentsline{toc}{section}{Generelle regler om Magi}
Disse regler gælder alle magier på alle tidspunkter hos alle professioner og der findes ingen undtagelser for normale spillere. Hvis der kommer monstre/plotkarakterer i spil, der kan det medføre en overtrædelse reglerne, men dette vil altid blive nævnt ved briefing.\\
\subsection*{Magiens elve bud}
\addcontentsline{toc}{subsection}{Magiens elve bud}
\begin{enumerate}
    \item Alle pegemagier og områdemagier har en maksimal rækkevidde på 5 meter.
    \item Der findes ingen magi der ikke bruger verbale komponenter, dette er ofte i form af en bøn eller en rite.
    \item Riter skal altid siges højt nok til, at dit mål kan høre det. Hører målet ikke riten, har de ret til at ignorere magien.
    \item Magiens kommando skal altid siges højt, så målet kan høre dig.
    \item Alle magier koster det dobbelte af deres niveau i mana at kaste. Dvs. en niveau 1 magi koster 2 mana, imens en niveau 3 magi koster 6 mana.
    \item Antallet af mana en magibruger har, er det samme som hans optjente XP, en magibruger kan maksimalt opnå 18 mana fra hans XP, samt eventuelt tilkøbt mana, skaffet gennem evner eller lignende. 
    \item Når en magibruger går på 0 LP, går han også automatisk på 0 mana.
    \item Hvis en magibruger kaster en magi, kan de kun holde den tilbage i 2 sekunder, før de skal kaste den. Kaster de den ikke, vil den ramme dem selv uanset effekt eller hvilke barrierer, der måtte beskytte dem fra magi.
    \item Alle magier kan bruges i og udenfor kamp. Vær opmærksom på, at når du forklarer en effekt til et offer, at i begge stadig in-game og derved kan tage skade osv. Hvis du er udenfor kamp eller skal kaste en magi på en person, som ligger ned, er det tilladt at bøje sig over personen og sige effekten lavt, således at du ikke forstyrre spillet.
    \item Alle magier går direkte igennem skjold, rustning og våben.
    \item Bruger du mere end 20 mana bør du tage kontakt til en arrangør.
\end{enumerate}

Der findes grundlæggende fire typer af magier:
\begin{itemize}
    \item Negativ magi
    \item Positiv magi
    \item Øjeblikkelig magi
    \item Passiv magi
\end{itemize}

{\large\textbf{Negativ magi}}\\
Magien påfører en negativ effekt på offeret, som varer i længere tid. En spiller kan kun være påvirket af en negativ magi af gangen. Skulle en spiller der allerede er påvirket af en negativ magi blive ramt af endnu en negativ magi, har den sidste magi ingen effekt.

{\large\textbf{Positiv magi}}\\
Magien påvfører en positiv effekt på en spiller, dette kan være i form af mere LP i længere tid eller et skjold. Skulle en spiller der allerede er påvirket af en positiv magi blive ramt af endnu en positiv magi, har den sidste magi ingen effekt.

{\large\textbf{Øjeblikkelig magi}}\\
Dette er magier der kun vare et øjeblik. Dette kan fx være helbredende magier eller magier der giver skade. Du kan være påvirket af uendelig mange øjeblikkelige magier, da disse vare i under et sekund.

{\large\textbf{Passiv magi}}\\
Magiske effekter som ikke kan fjernes. De vil altid være i effekt og koster ikke mana at bruge. De er også gældende hvis du er død.

Derudover findes der fire kategorier af magier:\\
{\large\textbf{Berøringsmagi}}\\
Når du kaster denne type magi skal du berører dit subjekt.

{\large\textbf{Pegemagi}}\\
Når du kaster denne type magi skal du pege på dit subjekt, der ikke må være længere væk end 5 meter.

{\large\textbf{Områdemagi}}\\
Denne magi er markeret med gryn (f.eks. havregryn). Denne magi aktiver når et eller flere subjekter krydser grynet.

{\large\textbf{Kastemagi}}\\
Denne magi kræver at du har en lille bold, rispose eller lignende. Denne skal du kaste på den du gerne vil ramme. Hvis du rammer med denne rispose så rammer magien også. Vær opmærksom på at selvom der ikke er nogen begrænsning på hvor langt du kan kaste denne bold, skal subjektet stadig kunne hører hvad din magi gør.

\subsection*{At kaste magi}
\addcontentsline{toc}{subsection}{At kaste magi}
Det vigtigste ved magikastning er at være velforberedt. Nogle magier har krav om visse ingredienser, som magikasteren selv skal medbringe. Hvis der er tale om en magi, hvor du skal ramme offeret med en genstand, f.eks. en skumbold, skal du sige effekten når genstanden rammer personen. Hvis genstanden rammer et sværd, skjold eller lignende, skal effekten stadig siges, da magien stadig har sin fulde effekt. Det er kun muligt at undgå magier, hvis offeret skal rammes, flytter sig. Dette kan f.eks. ikke lade sig gøre ved en pegemagi.

\chapter*{Niveau 3}
\addcontentsline{toc}{chapter}{Niveau 3}
Dine evner er vokset, og du styrer hvilke ånder der kommer til dig. Det er ikke længere dig som adlyder deres kald, men dem der kommer til dig for at opsøge din visdom.

\begin{table}[H]
    \centering
    \begin{tabular}{|p{0.50\textwidth}|p{0.25\textwidth}|}
    \rowcolor{cerulean!80}\hline
        Evne navn & Pris i XP \\\hline
        Læse/Skrive Magi & 1\\\hline
        Shamanmagi Niv. 3& 2\\\hline
        Ånde Ofring& 2\\\hline
        Åndens lærling Niv. 2& 2\\\hline
    \end{tabular}
\end{table}
\section*{Evne beskrivelse}
\addcontentsline{toc}{section}{Evne beskrivelse}

\input{mainstuff/Evner/Læse og skrive Magisk}

\subsection*{Shamanmagi Niv. 3} \addcontentsline{toc}{subsection}{Shamanmagi Niv. 3}
Du kan kalde dig selv Åndemager, og din kontrol over ånderne er kun blevet større. Denne nye titel giver dig mulighed for at få adgang til to Niv. 3 magier i de 3 stier du valgte fra niveau 2.\\
Du kan se hvilke magier du har adgang til, samt deres effekt i sektionen 'Magi'\\

\subsection*{Ånde Ofring} \addcontentsline{toc}{subsection}{Ånde Ofring}
Når shamanen laver en aktiv hyldest, kan han ofre en person til hans ånd. Denne må ikke tilhøre hans egen race, alle grønhuder tæller som at være den samme race. Laver han denne ofring vil han få dobbelt mana, når ritualet er færdigt.

\subsection*{Åndens Lærling Niv. 2} \addcontentsline{toc}{subsection}{Åndens Lærling Niv. 2}
Du får en valgfri evne fra samme profession som du har valgt i niveau 1. Denne evne må være fra niveau 1 til 3.\\
Når du vælger denne evne, skal du have de foregående niveauer for at tage evnen. F.eks. for at tage Alkymi niveau 2 skal du have Alkymi niveau 1.
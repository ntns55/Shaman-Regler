\chapter*{Niveau 1}
\addcontentsline{toc}{chapter}{Niveau 1}
Shamanen kommunikere med ånder. Du har fundet din vej ind i åndernes verden, og er klar til at starte din rejse igennem det mystiske og magiske.

\begin{table}[H]
    \centering
    \begin{tabular}{|p{0.50\textwidth}|p{0.25\textwidth}|}
    \rowcolor{cerulean!80}\hline
        Evne navn & Pris i XP \\\hline
        Brug af Ringbrynje & 2\\\hline
         Ekstra NK Niv. 1 & 1\\\hline
         Shamanmagi Niv. 1 & 1\\\hline
         Åndernes visdom & 2\\\hline
         Åndens lærling Niv. 1 & 1\\\hline
    \end{tabular}
\end{table}

\section*{Evne beskrivelse}
\addcontentsline{toc}{section}{Evne beskrivelse}

\subsection*{Brug af Ringbrynje}\addcontentsline{toc}{subsection}{Brug af Ringbrynje}
Du kan nu bruge ringbrynje som Shaman. 

\subsection*{Ekstra NK Niv. 1}
\addcontentsline{toc}{subsection}{Ekstra NK Niv. 1}
Du har et ekstra nævekamp.\\

\subsection*{Shamanmagi Niv. 1}
\addcontentsline{toc}{subsection}{Shamanmagi Niv. 1}
Du har længe prøvet at komme i kontakt med ånderne, og det er endelig lykkedes. Du kan nu kalde dig for en Åndekalder. Denne nye titel giver dig mulighed for at få adgang til alle Niv. 1 magier i de 4 stier.\\
Du kan se hvilke magier, du har adgang til, samt deres effekt i sektionen 'Magi'\\

\subsection*{Åndernes visdom}\addcontentsline{toc}{subsection}{Åndernes visdom}
Shamanen kan stille ånderne 2 spørgsmål pr. spilgang. Disse 2 spørgsmål skal stilles på samme tid. 

\subsection*{Åndens Lærling Niv. 1}\addcontentsline{toc}{subsection}{Åndens Lærling Niv. 1}
Når denne evne tages vil du gå i lære ved en ånd. Denne skal vælges med omhu, da den ikke kan vælges om senere.\\
Du skal du vælge en ikke magisk profession. Fra denne profession får du en valgfri evne fra niveau 1 eller 2. Når du vælger denne evne, skal du have de foregående niveauer for at tage evnen. F.eks. for at tage Alkymi niveau 2 skal du have Alkymi niveau 1.

